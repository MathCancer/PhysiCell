\documentclass[11point]{article}

\usepackage[letterpaper,margin=1in]{geometry}

\usepackage{amssymb}
\usepackage{amsmath}
\usepackage{bbm}
\usepackage{bm}


\newcommand{\beq}{\begin{eqnarray}}
\newcommand{\eeq}{\end{eqnarray}}
\renewcommand{\d}[1]{\:\mathrm{d}#1}
\renewcommand{\vec}[1]{\bm{#1}}
\newcommand{\grvec}[1]{\bm{#1}}


\newcommand{\one}{\mathbbm{1}}
\newcommand{\reals}{\mathbb{R}}

\newcommand{\df}[1]{\delta \left( #1 \right)}
\newcommand{\set}[1]{\left\{#1\right\}}

\newcommand{\var}[1]{\texttt{#1}}

\begin{document}
\section{Notation}
\begin{itemize}
\item
For any cell $i$ with center $\vec{x}_i$, let $R_i$ denote the 
region of space occupied by it. Assume that for any other cell 
$j$, that $R_i \cap R_j = \emptyset$. 
\item 
For any  computational 
mesh with voxels $\set{ \Omega }$ and corresponding 
volumes $\set{W}$, let $\rho( \Omega) $ denote the mean 
substrate density in voxel $\Omega$, and let 
$n(\Omega) = \int_\Omega \rho \d{V}$ denote the total
 amount of substrate in the voxel. 

Note that BioFVM tracks the mean substrate density in each voxel, 
so $\rho \equiv \rho( \Omega )$ throughout $\Omega$. 
\item 
For any voxel $\Omega_k$ with an index $k$, let $W_k$ denote 
its volume, define $\rho_k = \rho( \Omega_k )$, and 
define $n_k = n( \Omega_k)$. 

\item 
For any cell $i$ with center $\vec{x}_i$, let $\Omega_i$ denote the 
voxel containing cell $i$, with corresponding volume $W_i$. 
\item
Let $\one_i(\vec{x})$ be the characteristic  function for the cell, so that 
$\one_i(\vec{x}) =1$ inside the cell (inside $R_i$), and 
$\one_i(\vec{x}) = 0$ otherwise. 
\item 
Let $V_i = \int_{\reals^3} \one_i(\vec{x}) \d{V} = V_i$ be the total volume of cell $i$. 
\item 
For any cell $i$, let $N_i$ denote the \emph{internalized} total substrate. 

\end{itemize}

\section{Net extracellular substrate change 
due to the $i^\textrm{th}$ cell}
Note that in BioFVM the cells' contribution 
to changes in total substrate in any volume $\Omega$ is given by 
\beq
\frac{ \partial }{ \partial t} 
\int_\Omega \rho \d{V} & = &  
\sum_{\textrm{cells } i} 
\int_\Omega  \one_i(\vec{x})
\Bigl(  S_i \left( \rho^T_i - \rho \right)  - U_i \rho   \Bigr) \d{V}\\
& \approx & 
\sum_{\textrm{cells } i}
V_i \int_\Omega \df{ \vec{x} - \vec{x}_i }
\Bigl(  S_i \left( \rho^T_i - \rho \right)  - U_i \rho   \Bigr) \d{V}.
\eeq 

Now, let $\Omega = \Omega_i$ be the voxel containing $\vec{x}_i$ as 
defined above. Then assuming that only cell $i$ is in $\Omega_i$:  \beq
\frac{ d{n_i} }{ d{t} }  
= 
\frac{ \partial }{ \partial t} 
\int_{\Omega_i} \rho \d{V}
 & \approx &  
V_i 
\Bigl(  S_i \left( \rho^T_i - \rho(\vec{x}_i) \right)  - U_i \rho(\vec{x}_i)   \Bigr)  \\
& = & 
V_i 
\Bigl(  S_i \left( \rho^T_i - \rho_i \right)  - U_i \rho_i   \Bigr) .
\eeq 
(The case with multiple cells in a single computational voxel generalizes by performing this calculation separately for each cell contained in the voxel.)

Now, because $n_i = \rho_i W_i $, and asssuming $W_i$ is constant or changes very slowly compared to substrate densities, 
\beq
W_i \frac{ d{\rho_i} }{ d{t} }  & \approx &  
V_i 
\Bigl(  S_i \left( \rho^T_i - \rho_i \right)  - U_i \rho_i   \Bigr)   \\
\Longrightarrow 
\frac{d\rho_i }{dt} & \approx & 
\frac{ V_i }{W_i }
\Bigl( S_i \left( \rho_i^T - \rho_i \right) - U_i \rho_i \Bigr) 
\eeq 

\subsection{BioFVM implementation}
Now, let's apply a backward Euler scheme as in BioFVM, to determine the net change in total substrate in any time step with duration $\Delta t$: 
\beq
\frac{ \rho_i(t+\Delta t) - \rho_i(t)}{\Delta t} 
& \approx & 
\frac{ V_i }{W_i } 
\Bigl( S_i \left( \rho_i^T - \rho_i(t+\Delta t) \right) 
- U_i \rho_i( t + \Delta t ) \Bigr)\\
\Longrightarrow 
\rho_i( t+\Delta t) 
& \approx & 
\frac{\rho_i(t)  + c_1 }{c_2},
\eeq
where 
\beq
c_1 & = & \Delta t \frac{ V_i }{W_i } 
\left( S_i \rho_i^T \right) \\ 
c_2 & = & 1 + \Delta t \frac{ V_i }{W_i } 
\left( S_i + U_i \right) .
\eeq

This is the algorithm in 

\begin{center}\verb|void Basic_Agent::simulate_secretion_and_uptake( Microenvironment* pS, double dt )|
\end{center}

The constants $c_1$ and $c_2$ are set in 
\verb|void Basic_Agent::set_internal_uptake_constants( double dt )|. 

\subsection{Net extracellular substrate change}

Now, let's determine the change in total substrates in this implementation. First, 
\beq
n_i(t+\Delta t ) - n_i(t) & = & 
W_i \rho_i(t+\Delta t) - W_i \rho_i(t) \\
& = & 
W_i \left( \frac{\rho_i(t)+c_1}{c_2} - \rho_i(t) \right) \\ 
& = & 
W_i \left( \frac{\rho_i(t) + c_1  - c_2 \rho_i(t) }{c_2} \right) \\ 
& = & 
W_i \left( \frac{ (1-c_2) \rho_i(t) + c_1  }{c_2} \right) \\ 
\eeq
Notice that this can be calculated completely using constants that are already computed and used in BioFVM. 
\subsection{Algorithm}
We will use the following operations in the cell secretion/uptake function. (In the actual implementation, 
perform this on the entire vector of substrates, and use element-wise operations. i.e., Hadamard products and 
quotients.) 
\begin{enumerate}
\item 
\verb|change = 1 // 1|
\item 
\verb|change -= c2 // 1-c2|
\item 
\verb|change *= substrates // (1-c2)*rho|
\item 
\verb|change += c1 // (1-c2)*rho + c1|
\item 
\verb|change /= c2 // ((1-c2)*rho + c1)/c2|
\item 
\verb|change *= voxel_volume // W_i*((1-c2)*rho + c1)/c2|
\end{enumerate}
This is the net change in total substrates in $\Omega_i$. For conservation, 
the net chnage in cell $i$ is equal and opposite. Thus 
\begin{enumerate}
\setcounter{enumi}{6}
\item 
\verb|internalized_substrates -= change|
\end{enumerate}

\section{Additional option(s)}
If you set \verb|Basic_Agent::use_internal_densities_as_targets = true|, then whenever
the internal constants are changed, it sets 
\beq
\rho_i^* & = & \frac{N_i}{V_i}
\eeq
This criterion would be appropriate for non-active, diffusive secretion from the cell. 

Please note that 
if $\rho_i^* < \rho_i$, there is nothing in the mathematical form to prevent diffusion of the substrate 
back into the cell. If this is a concern, I suggest users manually test for that and set 
the secretion rates to zero accordingly. 

Future releases of PhysiCell may automate this testing, but we note that this test should be performed 
substrate-by-substrate. 

\section{Internal model}
Without an internal model, internalized substrate will reflect the total history of 
all uptaken substrates, or the sum tutoal fo all secreted substrates.  In particular, 
in the case of secretion, the internalized value will be negative to upload mass conservation. (No thing made inside, minus the secreted amount.) 

Users can provide their own internal model (e.g., for metabolomics), but we provide a ``sensible default.'' Inside the cell, we model: 
\beq
\frac{dN}{dt} & = & \overbrace{-u N}^\textrm{use} + \overbrace{c ( N^* -  n)}^\textrm{creation} \\
& = & 
-u \rho_I W + s( \rho_I^* - \rho_E ) W, 
\eeq
where $\rho_I$ is the internal density, and $c$, $u$, and $\rho_I^*$ are to be determined. We shall give the rationale for this form in the analysis below. 

To determine these parameters, let us consider the total amount of substrate in the cell: 
\beq
\frac{dN}{dt} & =& 
\overbrace{ U \rho_E W  }^\textrm{import} 
- 
\overbrace{ S \left( \rho_E^* - \rho_E \right) W }^\textrm{export} 
- 
\overbrace{ u \rho_I W }^\textrm{internal use} 
+ 
\overbrace{ c \left( \rho_I^* - \rho_E \right)W  }^\textrm{internal creation}
\eeq
Now, consider the case where there is uptake and use but no creation or secretion. Then 
\beq
\frac{dN}{dt} & = & 
U\rho_E W - u \rho_I W = \left( U \rho_E - u \rho_I \right) W. 
\eeq
In quasi-steady (or steady) conditions, we seek $u$ so that 
$\rho_I \approx \rho_E$. Notice that if $u = U$, then 
\beq
\frac{dN}{dt}  & = & U \left( \rho_E - \rho_I \right) W, 
\eeq
and so $\rho_I = \rho_E$ in quasi-steady condition. This model 
balances import with internal use. Moreover, if we balance all substrate in the 
environment (assuming without loss of generality only one uptaking cell): 
\beq
\frac{d}{dt} \left( n + N \right) & = &
- U \rho_E W + U \rho_E W - U \rho_I V = - U \rho_I W, 
\eeq
and over long times, $\rho_I \approx \rho_E$ , so we arrive at the 
normal situation from BioFVM where the overall loss rate is 
$- U \rho_E W$. 

Next, consider the case of only creation and export. In that case, 
\beq
\frac{dN}{dt} & = & 
-S ( \rho_E^* - \rho_E ) W + s ( \rho_I^* - \rho_E) W. 
\eeq
Suppose we follow our prior motivation and set $s = S$. Then 
\beq
\frac{dN}{dt} & = & 
S \left( - \rho_E^*  + \rho_E + \rho_I^* - \rho_E \right) W 
= S \left( -\rho_E^* + \rho_I^* \right)W. 
\eeq
If we choose $\rho_I^* = \rho_E^*$, then 
\beq
\frac{dN}{dt} &  = & 0
\eeq
and so we have successfully balanced creation and export. This the 
motivation for the functional form $s ( \rho_I^* - \rho_E)$ instead 
of the more obvious $s ( \rho_I^* - \rho_I)$. The biophysical interpretation 
is that the cell internally creates the substrate until the extenral density 
reaches the target value. 

\subsection{Summary:}
Returning now to the original notation where $\vec{\rho}$ is the vector of (extracellular) substrate densities, $\vec{N}$ is the vector of total internalized 
substrates, then we use (as an internal model), and 
\beq
\frac{d\vec{N}}{dt} & = & 
- \vec{u}\circ \vec{N} + \vec{c} \circ \left( \vec{\rho}^* - \vec{\rho}  \right)W
\eeq
and we set defaults: 
\beq
\vec{u} & = & \vec{U} \\
\vec{c} & = & \vec{S} 
\eeq

\section{Key cellular processes}
When a cell divides, it must distribute its internalized substrates to its daughter cells while maintaining conservation of mass. 

In PhysiCell 1.5.0, we do this by dividing the substrate by half in each of the daughter cells. 

When a cell dies and is removed by the simulation, multiple things could happen: for some substrates, it may make sense to 
remove them from the environment entirely, release it entirely at the time of lysis, or slowly release it back into the environment 
while the cell degrades. 

In PhysiCell, we opt for the simplest solution: release (some fraction) of internalized substrates when the basic agent (and hence cell) calls 
its destructor. Let $0 \le F \le 1$ denote the fraction of internalized 
substrates released at the time of death. 

We overwrite the density in the cell's voxel by first noting that there should be conservation of mass: 
\beq
n_i( t + \Delta t ) & = & 
n_i( t ) + F N_i (t)  \\ 
\Longrightarrow 
V_i \rho_i(t + \Delta t) & = &
V_i \rho_i(t) + F N_i(t) \\ 
\Longrightarrow 
\rho_i( t + \Delta t ) & = & 
\rho_i(t) + \frac{F}{V_i} N_i 
\eeq


\end{document}